% Use only LaTeX2e, calling the article.cls class and 12-point type.

\documentclass[12pt]{article}

% The following parameters seem to provide a reasonable page setup.

\topmargin 0.0cm
\oddsidemargin 0.2cm
\textwidth 16cm 
\textheight 21cm
\footskip 1.0cm

% Include your paper's title here

\title{AI Tetris - A CS3243 Project} 

\author
{Lau Kar Rui, A0155946U\\
Poh Jie, \\
Matilda, \\
Chee Wee, \\
\\
}

\date{25th March 2018}



% Start of document.
\begin{document} 

% Double-space the manuscript.

\baselineskip20pt

% Make the title.

\maketitle 


\section{Introduction}
	
	
\section{Utility Function}
We defined the utility function as a linear function $F$ with each feature $f_i \in Features$ having an assigned weight $w_i$, where $i = 1...n$, with $n$ being the number of features. Each $f_i$ derives a real value from a state $s$. The function is then defined as:
	$$F(s) = \sum_{i = 1}^{n} w_if_i(s)$$
The features $f_i$ used will be explained in depth in Section \ref{features}.
	
\section{Features Used} \label{features}
\begin{itemize}
  \item \textbf{RowsCleared}
  		- Number of rows cleared when the piece is put on the board
  \item \textbf{MaxHeightIncrease} 
  		- The maximum height increase when the piece is put on the board
  \item \textbf{AvgHeightIncrease} 
  		- The average height increase when the piece is put on the board
  \item \textbf{NumHoles} 
  		- Number of holes when the piece is put on the board; a hole is defined as an empty cell with a non-empty cell above it.
  \item \textbf{ColumnTransition}
  		- The total number of column transitions. A column transition occurs when an empty cell is adjacent to a filled cell on the same column and vice versa.
  \item \textbf{RowTransition}
  		- The total number of row transitions. A row transition occurs when an empty cell is adjacent to a filled cell on the same row and vice versa.
  \item \textbf{AbsHeightDiff}
  		- The height difference between all the columns.
\end{itemize}

\section{Training Function}

\section{Implementation}
\subsection{StateCopy}
In order to correctly apply the features, a new \texttt{StateCopy} class was created, extending the original \texttt{State} class, serving as a clean starting state to apply our features on in order to derive the feature value.

Extra variables like \texttt{currentRowsCleared} and \texttt{previousTop} is also added to the \texttt{StateCopy} class in order to obtain the information needed by various features such as the RowsCleared feature and the AverageHeight feature.

Using \texttt{StateCopy} also allows us to play moves without affecting the original state of the game.

\subsection{Feature}
Talk about the feature class here

\subsection{Learning Algo? Learner?}

\section{Scaling The Algorithm For Big Data}

\section{Results}

\section{Conclusion}



% Sample reference list for the various ways to reference.
\begin{thebibliography}{12}
\bibitem{latexcompanion} 
Michel Goossens, Frank Mittelbach, and Alexander Samarin. 
\textit{The \LaTeX\ Companion}. 
Addison-Wesley, Reading, Massachusetts, 1993.
 
\bibitem{einstein} 
Albert Einstein. 
\textit{Zur Elektrodynamik bewegter K{\"o}rper}. (German) 
[\textit{On the electrodynamics of moving bodies}]. 
Annalen der Physik, 322(10):891–921, 1905.
 
\bibitem{knuthwebsite} 
Knuth: Computers and Typesetting,
\\\texttt{http://www-cs-faculty.stanford.edu/\~{}uno/abcde.html}
\end{thebibliography}

\end{document}




















